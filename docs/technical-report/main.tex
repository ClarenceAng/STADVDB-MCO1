%%
%% The first command in your LaTeX source must be the \documentclass command.
\documentclass[sigconf, pbalance]{acmart}

\AtBeginDocument{%
  \providecommand\BibTeX{{%
    \normalfont B\kern-0.5em{\scshape i\kern-0.25em b}\kern-0.8em\TeX}}}

%%
%% Remove the ACM References statement and copyright notice.
\settopmatter{printacmref=false}
\renewcommand\footnotetextcopyrightpermission[1]{}

%% %% %% %% %%
%% Start of the actual paper.
\begin{document}

%%
%% Paper title.
\title{Data Warehousing and OLAP: A Case Study on the IMDb Dataset}

%%
%% Authors
\author{Clarence Ivan Ang}
\affiliation{%
  \institution{De La Salle University}
  \city{Manila}
  \country{Philippines}}
\email{clarence_ivan_ang@dlsu.edu.ph}

\author{Clive Jarel Ang}
\affiliation{%
  \institution{De La Salle University}
  \city{Manila}
  \country{Philippines}}
\email{clive_jarel_c_ang@dlsu.edu.ph}

\author{Malks Mogen David}
\affiliation{%
  \institution{De La Salle University}
  \city{Manila}
  \country{Philippines}}
\email{malks_david@dlsu.edu.ph}

\author{Rinaldo Adelrico Lee}
\affiliation{%
  \institution{De La Salle University}
  \city{Manila}
  \country{Philippines}}
\email{rinaldo_lee@dlsu.edu.ph}

%
%% This command allows the author to define a more concise list of authors' names for this purpose.
\renewcommand{\shortauthors}{Ang et al.}


%%
%% The abstract is a short summary of the work to be presented in the article.
\begin{abstract}
This paper presents the design and implementation of a data warehouse that integrates IMDb's non-commercial datasets with Box Office Mojo revenue data to enable comprehensive entertainment industry analytics. The warehouse employs a hybrid star-snowflake schema containing two fact tables—\texttt{FactRatingSnapshot} for audience ratings and \texttt{FactBoxOfficeRevenue} for box office performance—along with five dimension tables supporting analyses across titles, people, episodes, and time. A three-layer ETL pipeline processes approximately 10 GB of source data using MySQL-based extraction, SQL transformation operations including recursive CTEs and string parsing functions, and constraint-aware loading that maintains referential integrity. The system is accessed through a Next.js-based OLAP application providing five analytical modules for audience analytics, genre analysis, temporal trends, rating correlations, and revenue analysis. Performance optimization techniques including indexing strategies, query rewriting, and MySQL configuration tuning reduce typical query execution times from several minutes to under 10 seconds. The resulting platform demonstrates how dimensional modeling and ETL best practices can transform heterogeneous entertainment datasets into an integrated analytical environment supporting both industry decision-making and academic research.
\end{abstract}


%%
%% Keywords
\keywords{Data Warehouse, ETL, OLAP, Query Processing, Query Optimization}

%%
%% This command processes the author and affiliation and title information and builds the first part of the formatted document.
\maketitle

    \section{Introduction}

The entertainment industry generates vast amounts of data on audience reception and financial performance. IMDb, one of the largest film and television databases, catalogs over 11.9 million titles spanning movies, TV series, streaming content, and other media formats. In addition to this, Box Office Mojo tracks daily domestic box office revenue for thousands of theatrically released films. However, these datasets remain in their original operational formats, making it difficult to perform integrated analyses that combine critical reception with commercial success. This limitation restricts the ability of industry analysts and researchers to gain deeper insights into entertainment trends and patterns.

This project addresses this gap by designing a data warehouse that integrates IMDb's non-commercial datasets with Box Office Mojo revenue data. The warehouse transforms raw source files into a structured framework optimized for Online Analytical Processing (OLAP). OLAP refers to software tools that enable interactive examination of multidimensional data through operations like drill-down (navigating from summary to detail), roll-up (aggregating to higher levels), and slicing and dicing across dimensions \cite{codd1993olap}. The resulting system supports analyses of audience rating trends over time, correlations between IMDb ratings and box office performance, and genre-based popularity patterns.

The data warehouse was built using a hybrid star-snowflake architecture, also known as a starflake schema. This design balances query performance with data normalization by using bridge tables for many-to-many relationships while denormalizing attributes with bounded cardinality \cite{kimball2013datawarehouse}. The warehouse contains two fact tables—\texttt{FactRatingSnapshot} for audience ratings and \texttt{FactBoxOfficeRevenue} for box office data—along with five dimension tables including \texttt{DimTitle}, \texttt{DimPerson}, \texttt{DimEpisode}, and \texttt{DimDate}. Both fact tables share conformed dimensions, which would allow integrated analyses across different metrics.

An interactive web-based OLAP application provides a way to perform data analytics to the warehouse. Built using Next.js and React, the application offers five analytical modules: audience analytics, genre analysis, temporal trends, rating correlations, and revenue analysis. Users can perform drill-down operations (e.g., from annual ratings to monthly ratings for a specific genre), roll-up aggregations (e.g., from daily box office to quarterly revenue), and filtering across multiple dimensions. Overall, the application is intended for two primary user groups: entertainment industry analysts making business decisions and academic researchers investigating media trends.


    \section{Data Warehouse}

\subsection{The Dimensional Model}

The data warehouse was designed to transform IMDb’s non-commercial datasets into a structure suitable for business intelligence and analytics. The primary objective of the model is to support queries related to temporal trends across a wide variety of titles. Given these requirements, the schema follows a hybrid star–snowflake approach. Bridge tables are introduced to normalize and resolve many-to-many relationships (such as staff), while some dimensions are denormalized (such as genres) for the sake of query speed.

The grain of the warehouse is defined as \textit{one row per title per snapshot date}. This choice ensures that the fact table captures changes in user ratings over time, rather than overwriting them, thereby enabling analyses such as long-term rating trajectories, season-over-season comparisons, and historical popularity tracking. The overall schema is illustrated in Figure 1.

\subsection{Fact Tables}

The core of the schema is the \verb|FactRatingSnapshot| table. Each record corresponds to the state of IMDb ratings for a given title on a specific snapshot date. The table contains two central measures: (i) the weighted average rating and (ii) the total number of votes. In addition, surrogate keys link the fact table to its associated dimensions (or to itself in the case of \verb|parentTitleID|, which will be discussed later). Audit columns such as \verb|dateCreated| and \verb|dateModified| are also added to store ETL metadata and serve as a compliance trail.

From a modeling perspective, this table is implemented as a periodic snapshot fact table (Kimball \& Ross, 2013). While a transaction fact table would be ideal for the business requirements of the data warehouse, IMDb's dataset only provides the aggregated rating since it uses weighted averages calculated through undisclosed algorithms. Additionally, an accumulating snapshot table is not a viable option because it assumes a milestone-based process, which is practically nonexistent in the context of the business requirements. The decision was then made to model \verb|FactRatingSnapshot| as a periodic snapshot to preserve the temporal characteristic of the ratings data. Ratings and votes change continuously, and a design that overwrites the latest values would discard the historical trail necessary to analyze trends. For these reasons, the use of a periodic snapshot fact table is the most suitable method to capture the evolving state of these aggregated metrics.

\subsection{Dimensions and Hierarchies}

A key strength of any dimensional model lies in its ability to support OLAP operations (such as drill-down and roll-up). Each dimension was therefore designed with hierarchies that allow analysts to navigate from coarse to fine levels of granularity. This would also ensure that the schema aligns directly with the intended business intelligence applications.

\begin{itemize}
    \item \textbf{DimTitle.} This is the central descriptive dimension for all the titles in the dataset (which includes movies, series, episodes, shorts, and other formats). Attributes include identifiers, titleType, primaryTitle, originalTitle, adult flag, runtime, and production years. A self-referencing hierarchy is defined through the parentTitleID, which connects episodes to their series. This hierarchy enables drilling down from series → season → episode, or rolling up in the opposite direction, which is crucial for trend analyses such as “average rating per season” or “consistency across episodes.”
    \item \textbf{DimEpisode.} While the series–episode relationship could be modeled solely within DimTitle, DimEpisode was introduced as a complementary dimension to capture episode-specific attributes such as seasonNumber and episodeNumber. This separation makes episode-level drill-downs semantically clearer and query-friendly: analysts can filter directly by season and episode numbers rather than relying exclusively on title metadata. DimEpisode thus works in tandem with DimTitle, offering two parallel but consistent ways of expressing hierarchy.
    \item \textbf{DimPerson.} Contributors such as directors, actors, and writers are modeled here, with attributes including primaryName, birthYear, deathYear, and primaryProfession. The bridge structure between titles and persons captures the fact that multiple people can contribute to a title, and individuals can span many titles. Within DimPerson, a profession-based hierarchy is implicit (e.g., roll-up from individual actors to all actors), which supports analyses like “average ratings by director” or “titles per actor.”
    \item \textbf{DimDate.} This dimension provides the calendar framework for both release and snapshot analyses. It implements the conventional hierarchy of Year → Quarter → Month → Day. This hierarchy enables drill-down from a decade to specific years or even days, and roll-up to higher-order periods. For example, trends in average ratings can be tracked by decade, narrowed down to individual years, or analyzed across quarters to detect seasonality.
\end{itemize}

\subsection{Issues Encountered}

\begin{table}
  \caption{The different title types and their count.}
  \label{tab:title-types}
  \begin{tabular}{cc}
    \toprule
    \textbf{titleType} & \textbf{COUNT(*)} \\
    \midrule
    short           & 1,084,187 \\
    movie           & 726,145   \\
    tvShort         & 10,764    \\
    tvMovie         & 152,398   \\
    tvEpisode       & 9,181,254 \\
    tvSeries        & 287,817   \\
    tvMiniSeries    & 65,091    \\
    tvSpecial       & 54,354    \\
    video           & 316,361   \\
    videoGame       & 45,244    \\
    tvPilot         & 1         \\
    \bottomrule
  \end{tabular}
\end{table}

    \section{ETL Script}

The ETL pipeline integrates data from IMDb's non-commercial datasets and Box Office Mojo revenue data into the analytical warehouse. The process follows Kimball's three-layer architecture \cite{kimball2004etl}: extraction from source files, transformation through a staging area, and loading into the dimensional warehouse.

\subsection{Data Volume and Loading Challenges}

\subsubsection{Source Data Characteristics}

IMDb's dataset comprises six compressed TSV files totaling approximately 1.37 GB compressed, with the largest file (\texttt{title.principals.tsv}) containing over 94 million rows. Box Office Mojo contributes 22 MB of daily box office revenue data. The populated warehouse occupies 13 GB, reflecting a good amount of storage overhead from denormalization, surrogate keys, and indexes.

\begin{table}[h]
  \centering
  \caption{IMDb and Box Office Mojo data source volumes.}
  \label{tab:etl-volumes}
  \begin{tabular}{lr}
    \toprule
    \textbf{Dataset} & \textbf{File Size (Compressed)} \\
    \midrule
    name.basics.tsv.gz       & 291.99 MB \\
    title.basics.tsv.gz      & 212.31 MB \\
    title.crew.tsv.gz        & 78.29 MB \\
    title.episode.tsv.gz     & 51.01 MB \\
    title.principals.tsv.gz  & 736.18 MB \\
    title.ratings.tsv.gz     & 8.20 MB \\
    \midrule
    \textbf{IMDb Total}      & \textbf{1.37 GB} \\
    \midrule
    box\_office\_revenue.tsv & 22.16 MB \\
    \midrule
    \textbf{Grand Total}     & \textbf{1.40 GB} \\
    \bottomrule
  \end{tabular}
\end{table}

\subsubsection{Performance Optimization and Configuration Tuning}

Initial ETL attempts resulted in quite a number of timeout errors, memory exhaustion, and recursion limit failures. To address these issues, the MySQL configuration had to be tuned:

\begin{itemize}
    \item \texttt{innodb\_buffer\_pool\_size = 8GB}: Increased from the default to cache dimension tables and indexes
    \item \texttt{max\_allowed\_packet = 2GB}: Expanded from 16 MB to accommodate large recursive CTE result sets
    \item \texttt{cte\_max\_recursion\_depth = 5000}: Raised from 1,000 to handle deeply nested director/writer lists
    \item Network timeouts: Extended to 8 hours for long-running bulk transformations
\end{itemize}

\subsubsection{Incremental vs. Full Refresh Strategy}

The current ETL implementation uses a \textit{full refresh} strategy in which each execution drops and recreates all staging and warehouse schemas, then repopulates from the source files. This approach, while operationally simple and ensures consistent data, is unsustainable for production environments where IMDb updates daily. A full refresh ETL cycle requires approximately 2 hours end-to-end, making daily incremental updates preferable when deploying it in practical applications.

While not done due to the scope of MCO1, future iterations could use \textit{incremental loading} and Change Data Capture (CDC) techniques. IMDb publishes daily differential files identifying added, modified, and deleted records. By processing only these deltas and implementing Slowly Changing Dimension Type 2 tracking (preserving historical attribute versions), the warehouse can be updated in near real-time without experiencing the penalties of a full refresh.

\subsection{Extraction Process}

Data is sourced from IMDb \cite{imdb2025datasets} and Box Office Mojo \cite{boxofficemojo2025}. IMDb provides seven TSV files: \texttt{name.basics} (12M+ personnel records), \texttt{title.basics} (11.9M titles), \texttt{title.crew} (director/writer attributions), \texttt{title.episode} (series hierarchy), \texttt{title.principals} (principal cast/crew), and \texttt{title.ratings} (aggregated ratings), and \texttt{title.akas} (dropped as discussed in Section 2). Box Office Mojo also provides data on daily domestic box office revenue obtained via web scraping.

Files are manually downloaded, decompressed from \texttt{.gz} format, and placed in \texttt{C:/ProgramData/MySQL/MySQL Server 8.0/Uploads/} to satisfy MySQL's \texttt{secure\_file\_priv} security restriction. The staging area uses \texttt{LOAD DATA INFILE} for high-performance bulk loading, directly loading the rows into the tables without SQL parsing overhead. The datasets were extracted on September 15, 2025.

\subsection{Transformation Process}

Transformations are implemented as SQL-based operations within MySQL to minimize data movement and use database-native optimization. Key transformation patterns include:

\subsubsection{String Parsing for Multi-Valued Attributes}

As stated earlier, IMDb encodes multi-valued attributes as comma-separated lists. One solution to address that is for fixed-width attributes (genres, professions) to be parsed using \texttt{SUBSTRING\_INDEX} into discrete columns.

Meanwhile, variable-width attributes (e.g., directors and writers) require recursive CTEs to convert the lists into individual rows. The recursive pattern iteratively extracts identifiers until the list is exhausted, transforming a single row with $N$ directors into $N$ discrete rows. This decomposition accounts for approximately [30\%] of the transformation time, with recursion depths occasionally exceeding 1,000 iterations for titles with a large cardinality of directors and/or writers.

\subsubsection{Data Type Conversions}

Additionally, IMDb's string encoded data requires a number of explicit conversions: \texttt{isAdult} flags are converted using \texttt{CASE WHEN isAdult = '1' THEN TRUE}, and IMDb's \texttt{\N} NULL markers are handled via \texttt{LOAD DATA INFILE} configuration. Numerical attributes like ratings use \texttt{DECIMAL(3,1)} to match IMDb's precision; while revenue uses \texttt{DECIMAL(12,2)} for box office figures.

\subsubsection{Surrogate Key Generation}

The warehouse uses \texttt{AUTO\_INCREMENT} surrogate keys (\texttt{titleID}, \texttt{personID}) for join performance (4-byte integers vs. 20-character strings) and immutability. More importantly, natural keys are preserved as \texttt{UNIQUE} columns (\texttt{tconst}, \texttt{nconst}) to maintain a link with the source data and support incremental ETL. While populating the fact tables, joins would reference the natural keys (\texttt{JOIN DimTitle t ON t.tconst = r.tconst}), which MySQL does efficiently via unique indexes.

The \texttt{AUTO\_INCREMENT} key generation during bulk inserts shows a relatively predictable behavior: keys are assigned sequentially in insertion order. No key collisions or gaps (beyond those caused by failed insertions) were observed, though transaction rollbacks can introduce non-contiguous sequences. This is an acceptable trade-off especially considering the semantics of ETL and negligibility in the overall scheme of things.

\subsubsection{Date Dimension Population Strategy}

\texttt{DimDate} is populated on-demand rather than pre-populated with a date range. The ETL script inserts dates referenced by box office revenue records, the current execution date (\texttt{CURDATE()}), and the subsequent date (\texttt{CURDATE() + 1}) to accommodate snapshot operations.

The \texttt{INSERT IGNORE} clause prevents duplicate key errors when dates already exist (e.g., box office revenue spanning multiple years overlaps with snapshot dates). This on-demand strategy minimizes \texttt{DimDate} size—only dates actively referenced in fact tables are materialized—at the cost of requiring ETL updates when new date ranges emerge (e.g., future box office releases).

Alternative strategies include pre-populating a 200-year range (1900–2100) to make sure that all conceivable dates exist upfront. This trades a modest storage cost (approximately 73,000 rows) for ETL simplicity. However, the on-demand approach aligns with the warehouse's current operational context (historical data analysis rather than prospective forecasting) and was therefore kept in the end.

\subsection{Loading Process and Constraints}

The ETL uses a three-layer system: source files, staging schemas (\texttt{imdb}, \texttt{boxofficemojo}), and the data warehouse. The staging area serves as an intermediate landing zone where raw TSV data is bulk-loaded with minimal transformations. However, the primary disadvantage to this approach is storage redundancy: staging tables duplicate source file content, consuming additional disk space. For multi-gigabyte datasets, this overhead is non-trivial but acceptable given the analytical and operational benefits that it would provide.

Foreign key and unique constraints are temporarily disabled during bulk loading (\texttt{SET FOREIGN\_KEY\_CHECKS = 0; SET UNIQUE\_CHECKS = 0;}) to accelerate insertion, with defensive \texttt{WHERE} clauses filtering invalid references before insertion. Constraints are re-enabled post-load for validation.

Dimensions are loaded before facts following dependency order: 

\begin{enumerate}
    \item \texttt{DimPerson}, \texttt{DimTitle}: Independent dimensions with no internal dependencies
    \item \texttt{BridgeTitlePerson}, \texttt{DimCrew}: Depend on \texttt{DimPerson} and \texttt{DimTitle}
    \item \texttt{DimEpisode}: Depends on \texttt{DimTitle} (both parent and child references)
    \item \texttt{DimDate}: Populated from box office and snapshot dates
    \item \texttt{FactRatingSnapshot}, \texttt{FactBoxOfficeRevenue}: Depend on all dimensions
\end{enumerate}

A notable edge case: \texttt{DimEpisode} references \texttt{DimTitle} twice (parent and child), creating a potential circular dependency if episodes could themselves be parents. IMDb's data model prohibits this (because only series can be parents), eliminating the dependency. The ETL script's \texttt{WHERE te.tconst IS NOT NULL AND te.parentTconst IS NOT NULL} filter ensures both references are valid before insertion.

\subsection{Issues Encountered and Resolutions}

\subsubsection{Performance Bottlenecks}

As expected, the recursive CTE execution for parsing director/writer served as the primary bottleneck, accounting for 30\% of the transformation time. Some attemps at optimizing the query included increasing \texttt{cte\_max\_recursion\_depth} to 5,000 (preventing errors but not accelerating execution) and indexing source tables on \texttt{title\_crew(tconst, directors, writers)}. The latency was ultimately accepted as unavoidable given data volume.

\subsubsection{Memory Exhaustion and Configuration}

Early executions triggered out-of-memory errors during large \texttt{INSERT INTO ... SELECT} operations. Tuning \texttt{innodb\_buffer\_pool\_size} to 8 GB resolved this, providing sufficient cache for dimension tables. It was noted that CPU utilization peaked at 99\% during transformation phases, indicating that the workload is I/O-bound (limited by disk read/write speeds) rather than CPU-bound. Upgrading to SSD storage or increasing buffer pool size further could yield additional performance gains.

\subsubsection{Box Office Mojo Title Matching}

Combining Box Office Mojo revenue data with IMDb titles required finding a way to resolve the title identifiers across the different datasets. Box Office Mojo uses proprietary internal IDs, while IMDb uses \texttt{tconst} identifiers. The \texttt{BoxOfficeMojoIds} mapping table bridges this gap, associating Box Office Mojo IDs with IMDb \texttt{tconst} values.

Box Office Mojo, originally an independent film revenue tracking website, was acquired by IMDb in 2008. As part of the integration, IMDb’s unique identifier (\texttt{tconst}) was incorporated into Box Office Mojo’s URLs, which created a direct correspondence between entries on both platforms. Consequently, the matching process simply required mapping each \texttt{boxofficemojo\_id} to its associated \texttt{tconst}. This approach resulted in complete coverage: every Box Office Mojo title was successfully linked to its IMDb record.

    \section{OLAP Application}

\subsection{Purpose of the Application}

The main purpose of our OLAP application is to help investors or companies gain valuable insight into consistent directors who have performed well in terms of both ratings and revenue. The application also explores movies that may have scored low in ratings but still achieved a high revenue for its rating. Additionally, it provides data on movies of certain genres across specific months or years to identify which genres are thriving or underperforming. This information helps users determine the best time of year to release certain genres and gives them a strong starting point for further research into performance trends.

\subsection{Decision-Making or Analytical Tasks}

The application supports decision-making and analytical tasks such as identifying high profile directors, evaluating the relationship between movie ratings and revenue, and analyzing genre performance over time. It also assists users in discovering patterns between audience reception and profitability through performing a Spearman correlation test between ratings and revenue. These results enable investors and companies to make data-decisions on production strategies, release timing, and deciding on a genre.

\subsection{Analytical Reports for Each Query}

As a foreword: technically, slicing is present in all the queries, since we’re only ever considering movies in our analyses (which is a slice of the overall dataset). In hindsight, maybe we should've dropped non-movie data, but it was still something we were considering to analyze; we just didn't end up doing anything with episodal entries.

\paragraph{Query 1}
This query retrieves the directors with relatively consistent high-grossing movies. By consistent we mean they’ve made at least 3 movies and have one of the highest average per-movie revenues. This query uses dicing to select the movies along the DimTitle dimension and their corresponding directors in the DimCrew dimension.

As a primary optimization, aside from the structure of the schema, you can see that most of the filtering was done on their respective tables prior to joining. This is also the case for most of the other queries, where the titleType = ‘movie’ filter is performed before joining with other tables. An inspection of the dataset reveals that this clause reduces the number of rows returned from DimTitle from around 1 million to a few thousand (apparently most of the IMDB data entries we have mapped are episodes, which should make sense).

Another slice is done through the DimCrew dimension, which serves to further optimize the behavior of the query (since this query is only interested in looking at director performance).

\begin{lstlisting}
WITH MovieGrossRevenues(titleID, totalGrossRevenue) 
AS (
    SELECT 
        fbor.titleID, 
        SUM(fbor.grossRevenue) as totalGrossRevenue
    FROM FactBoxOfficeRevenue fbor
    JOIN ( 
        SELECT dt.titleID
        FROM DimTitle dt
        WHERE dt.titleType = 'movie'
    ) mov ON mov.titleID = fbor.titleID
    GROUP BY fbor.titleID
)
    SELECT 
        dp.primaryName, 
        MIN(mgr.totalGrossRevenue) / 1000000 
            AS minPerMovieRevenueInMillions, 
        MAX(mgr.totalGrossRevenue) / 1000000 
            AS maxPerMovieRevenueInMillions, 
        AVG(mgr.totalGrossRevenue) / 1000000 
            AS avgPerMovieRevenueInMillions, 
        COUNT(mgr.totalGrossRevenue) AS totalMovies
    FROM MovieGrossRevenues mgr
    JOIN (
        SELECT personID, titleID
        FROM DimCrew dc
        WHERE dc.position = 'director'
    ) dir ON dir.titleID = mgr.titleID
    JOIN DimPerson dp ON dp.personID = dir.personID
    GROUP BY dir.personID
    HAVING totalMovies >= 3
    ORDER BY avgPerMovieRevenueInMillions DESC;
\end{lstlisting}

\paragraph{Query 2}
This query summarizes the earnings for the movies of a particular genre across different time intervals. We can consider some of the operations here to represent a dicing of the dataset, since we’re filtering across multiple columns (titleType along the title dimension, and different date fields along the date dimension). The query also constitutes a rollup since we’re performing an aggregation across decreasing levels of granularity for the date dimension.

Similar to the other queries, we filter along the DimTitle dimension before joining the fact tables. This time, since we’re including the genre to perform the filter, we also have way less rows to deal with during the join. Additionally, because we’ve indexed the date dimension along date IDs, joining and querying along this dimension is also way more efficient.
\begin{lstlisting}
WITH GenreMoviesRevenues(
    titleID, 
    primaryTitle, 
    totalGrossRevenueDuringPeriodForMovie, 
    year, 
    month
) AS (
    WITH GenreMovies(titleID, primaryTitle) 
    AS (
        SELECT titleID, primaryTitle
        FROM DimTitle
        WHERE titleType = 'movie'
        AND (
            genre1 = 'horror'
            OR genre2 = 'horror'
            OR genre3 = 'horror'
        )
    )
    SELECT 
        gm.titleID, 
        gm.primaryTitle, 
        MAX(fbor.grossRevenueToDate) 
            AS totalGrossRevenueDuringPeriodForMovie, 
        dd.year, 
        dd.month
    FROM GenreMovies gm
    JOIN FactBoxOfficeRevenue fbor
    ON fbor.titleID = gm.titleID
    JOIN DimDate dd
    ON dd.dateID = fbor.revenueRecordDateID
    GROUP BY 
        gm.titleID, 
        gm.primaryTitle, 
        dd.year, 
        dd.month
)
    SELECT 
        SUM(totalGrossRevenueDuringPeriodForMovie) 
            AS totalGrossRevenuesDuringPeriodForAllMovies, 
        year, 
        month
    FROM GenreMoviesRevenues
    GROUP BY year, month
    WITH ROLLUP;
\end{lstlisting}

\paragraph{Query 3}
This query determines directors with a fairly consistent ranking streak across at least 20 movies. We sort the directors according to their rating in descending order. Again this query constitutes dicing since we’re filtering across multiple dimensions.

We perform a sequence of slices to arrive at the final dice of data we need. By doing these filters sequentially instead of after joining all three tables, we minimize the complexity of the resulting query.

\begin{lstlisting}
SELECT 
    dp.primaryName, 
    fff.finalAverageRating, 
    fff.movieCount
FROM DimPerson dp
JOIN (
    WITH DirMovs(titleID, personID, averageRating) 
    AS (
        SELECT 
            frs.titleID, 
            m.personID, 
            AVG(frs.averageRating) 
        FROM FactRatingSnapshot frs
        JOIN (
            SELECT dt.titleID, dir.personID
            FROM DimTitle dt
            JOIN (
                SELECT titleID, personID
                FROM DimCrew dc 
                WHERE position = 'director'
            ) dir
            ON dt.titleID = dir.titleID
            WHERE dt.titleType = 'movie'
        ) m
        ON m.titleID = frs.titleID
        GROUP BY frs.titleID, m.personID
    )
        SELECT 
            personID, 
            AVG(averageRating) 
                AS finalAverageRating, 
            COUNT(titleID) 
                AS movieCount
        FROM DirMovs
        GROUP BY personID
        HAVING movieCount >= 20
) fff
ON fff.personID = dp.personID
ORDER BY fff.finalAverageRating DESC;
\end{lstlisting}

\paragraph{Query 4} 
This query looks for when a movie earned the most in a day, since the date of its release. It does this by first determining the max gross revenue (in a day) of each movie, and filters the fact table for that specific movie and revenue value. I guess this query essentially slices the revenue fact table based on the maximum revenue of each movie, and then joins the date table for more info. This was the most efficient implementation we could think of, but I'm not entirely sure if it's optimal. Nevertheless, another slight optimization performed here is the division by 1000000, which is performed only after aggregation instead of prior.
\begin{lstlisting}
WITH GrossRevenueByDate(
    titleID, 
    grossRevenue, 
    daysSinceRelease, 
    year, 
    month, 
    day
) AS (
    SELECT 
        dt.titleID, 
        fbor.grossRevenue, 
        fbor.daysSinceRelease, 
        dd.year, 
        dd.month, 
        dd.day
    FROM (
        SELECT titleID
        FROM DimTitle
        WHERE titleType = 'movie'
    ) dt
    JOIN FactBoxOfficeRevenue fbor 
    ON dt.titleID = fbor.titleID
    JOIN DimDate dd 
    ON dd.dateID = fbor.revenueRecordDateID
)
    SELECT 
        AVG(grd1.grossRevenue) / 1000000 
            AS grossRevInMillions, 
        grd1.daysSinceRelease
    FROM GrossRevenueByDate grd1
    JOIN (
        SELECT 
            MAX(grossRevenue) AS maxGrossRevenue, 
            titleID
        FROM GrossRevenueByDate
        GROUP BY titleID
    ) grd2
    ON grd1.titleID = grd2.titleID
    WHERE grd1.grossRevenue = grd2.maxGrossRevenue
    AND daysSinceRelease IS NOT NULL
    GROUP BY grd1.daysSinceRelease
    ORDER BY grd1.daysSinceRelease;
\end{lstlisting}

\paragraph{Query 5}
This query tries to look at the worst rated movies with the highest revenues. Because we’re filtering on multiple columns, we’re effectively dicing the dataset. This query also benefits from additional indices on the rating and revenue columns of their respective dimension tables, (although the next query was the primary reason for adding these). Similar to the previous query, the division by 1000000 is performed only after aggregation instead of prior.
\begin{lstlisting}
SELECT fbor.titleID,
    lrm.primaryTitle,
    SUM(grossRevenue) AS cumulativeRevenue,
    AVG(lrm.averageRating) AS avgRating
FROM FactBoxOfficeRevenue fbor
JOIN (
    SELECT 
        dt.titleID, 
        dt.primaryTitle, 
        dt.startYear, 
        lr.averageRating
    FROM DimTitle dt
    JOIN (
        SELECT titleID, averageRating
        FROM FactRatingSnapshot
        WHERE averageRating < 4.0
    ) lr ON lr.titleID = dt.titleID
    WHERE dt.titleType = 'movie'
) lrm ON lrm.titleID = fbor.titleID
JOIN DimDate dd 
ON dd.dateID = fbor.revenueRecordDateID
GROUP BY titleID, lrm.primaryTitle
ORDER BY cumulativeRevenue DESC
LIMIT ? OFFSET ?;
\end{lstlisting}


\paragraph{Query 6}
Our statistical query! A masterpiece I would say! This computes the spearman correlation between the revenue and the rating of a specific genre of movies. It does this by first generating a new column that holds the ranking of each row with regard to both rating and revenue. It then computes the coefficient through the formula based on those two ranks. To speed up the computation of this specific query, indexes were created for these two columns in their respective tables. This was the primary reason for creating these additional indices outside of the primary keys of the schema.
\begin{lstlisting}
WITH N(total, count) AS (
  WITH Deltas(deltaSquared) AS (
    WITH Ranks(revenueRank, ratingRank) AS (
      WITH Movies(titleID) AS (
        SELECT dt.titleID
        FROM DimTitle dt
        WHERE dt.titleType = 'movie'
        AND (
          dt.genre1 = ?
          OR dt.genre2 = ?
          OR dt.genre3 = ?
        )
      )
        SELECT 
          ROW_NUMBER() OVER (
            ORDER BY 
              tgr.totalGrossRevenueToDateInMillions DESC
          ) AS revenueRank,
          ROW_NUMBER() OVER (
            ORDER BY tr.rating DESC
          ) AS ratingRank
        FROM (
          SELECT 
            m.titleID, 
            MAX(grossRevenueToDate) / 1000000 
              AS totalGrossRevenueToDateInMillions
          FROM FactBoxOfficeRevenue fbor
          JOIN Movies m
          ON m.titleID = fbor.titleID 
          GROUP BY m.titleID
        ) tgr
        JOIN (
          SELECT 
            m.titleID, 
            MAX(frs.averageRating) 
              AS rating
          FROM FactRatingSnapshot frs 
          JOIN Movies m
          ON m.titleID = frs.titleID 
          GROUP BY m.titleID
        ) tr 
        ON tgr.titleID = tr.titleID
        ORDER BY revenueRank
    )
      SELECT 
        revenueRank * revenueRank + 
        ratingRank * ratingRank - 
        2 * revenueRank * ratingRank 
          AS deltaSquared
      FROM Ranks
  )
    SELECT 
      6 * SUM(deltaSquared) AS total, 
      COUNT(*) AS count FROM Deltas
)
  SELECT 
    total / (
      count * count * count - 
      count
    ) AS spearman
  FROM N;
\end{lstlisting}

\paragraph{Query 7}
This query helps us determine whether or not revenue has seasonal patterns throughout the duration of a year. We plot the different revenues as line charts segregated by year, superimposed on top of each other within the same graph. This query uses a drill-down along the date dimension and performs a group by along the year, month, and day dimensions. A slight optimization performed here (similar to some of the previous queries) is the division by 1000000, which is performed only after aggregation instead of prior.
\begin{lstlisting}
WITH Movies(titleID) 
AS ( 
    SELECT dt.titleID
    FROM DimTitle dt
    WHERE dt.titleType = 'movie'
)
    SELECT 
        SUM(fbor.grossRevenue) / 1000000 
            AS totalRevenueThatDayInMillions, 
        dd.year, 
        dd.month, 
        dd.day
    FROM Movies m
    JOIN FactBoxOfficeRevenue fbor 
    ON fbor.titleID = m.titleID
    JOIN DimDate dd 
    ON dd.dateID = fbor.revenueRecordDateID
    WHERE dd.year = ?
    GROUP BY dd.year, dd.month, dd.day
    ORDER BY dd.year, dd.month, dd.day;
\end{lstlisting}


    \section{Query Processing and Optimization}

\subsection{What is Query Optimization?}

Query optimization is the act of reducing the runtime of queries in your database. This may be done through database design, indexes, query restructuring, or through hardware optimizations. It has become a topic that it is necessary to learn in today's world as in our data-driven society, it is imperative we are able to analyze data quickly and efficiently in the shortest time possible. Through query optimization, we can reduce runtime of queries from hours to mere minutes or even seconds. It is a concept that is important since as our world continues to collect more data, it is our job to make sure that we are able to use it and turn it into valuable information as quickly as possible.

\subsection{Query Optimization Strategies}

\subsubsection{Database Design and Normalization}

Proper database design and normalization play a major role in designing an OLAP application, as both the star and snowflake schemas have their own advantages and disadvantages. In a star schema, tables are denormalized, meaning that data is stored in a more straightforward and flattened structure. This design allows for faster query performance since fewer joins are required, but it can lead to increased storage requirements. On the other hand, the snowflake schema is normalized, meaning that there are much more tables that are stored to encourage joins, thus reducing the query runtime. However the benefit of this is that it uses far less storage compared to a star schema, making it beneficial as data warehouses are known to have large amounts of data that exceeds gigabytes of storage. For the purposes of our OLAP application, we had used the third type of schema, the starflake schema, in which only certain tables are normalized as it balances the storage and performance of queries, bringing out the best of both worlds \cite{ibm2021datawarehouse,kimball2013datawarehouse}. \\

As aforementioned, there are concepts within the database such as genre which has a very limited domain, which makes it rather unjustifiable to have its own table, rather than having to store 3 genre IDs that still have to be connected and joined to a different table. Decisions such as this are important to be captured through data profiling in order to make informed decisions to optimize the database even further. \\

\begin{figure}[H]
    \centering
    \includegraphics[width=0.8\linewidth]{images/schema.png}
    \caption{Schema for the OLAP Application}
    \label{fig:your_label}
\end{figure}

\subsubsection{Indexes}

Indexes are used in order to query results efficiently and is an essential tool in improving query performance. In the purposes of our OLAP application that analyzes rating, revenue, and genres, it was imperative that we create secondary indexes on these columns to decrease the runtime significantly. 

\begin{lstlisting}
CREATE INDEX index_position
ON DimCrew (position)

CREATE INDEX index_genre
ON DimTitle (genre1, genre2, genre3)

CREATE INDEX index_title
ON DimTitle (titleType)

CREATE INDEX index_rating
ON FactRatingSnapshot (averageRating)

CREATE INDEX index_revenue
ON FactBoxOfficeRevenue (grossRevenueToDate)
\end{lstlisting}

\subsubsection{Query Restructuring}

The queries we had used in our application had numerous optimizations, restructuring, and formatting for readability. CTEs were used throughout majority of the queries for formatting alongside the inner queries being the most selective to return the least amount of rows as possible. Doing this decreases the query runtime and allows us to retrieve thousands to tens of thousands in a sea of tens of millions of rows in mere seconds. Additionally, operations such as only selecting required columns, aggregates and early filtering proved to be effective in further decreasing the runtime to strive for the most optimized query runtime as possible.

\begin{lstlisting}
WITH Movies(titleID) AS ( 
        SELECT dt.titleID
        FROM DimTitle dt
        WHERE dt.titleType = 'movie'
    )
SELECT 
    SUM(fbor.grossRevenue) / 1000000 
        AS totalRevenueThatDayInMillions, 
    dd.year, 
    dd.month, 
    dd.day
FROM Movies m
JOIN FactBoxOfficeRevenue fbor 
ON fbor.titleID = m.titleID
JOIN DimDate dd 
ON dd.dateID = fbor.revenueRecordDateID
WHERE dd.year = "2024"
GROUP BY dd.year, dd.month, dd.day
ORDER BY dd.year, dd.month, dd.day;
\end{lstlisting}

Such example shows that the table for Movies is much more readable since it is contained in the CTE rather than being  stored in a subquery.

\subsubsection{Optimization at Hardware Level}

Alongside database design, indexes, and query restructuring, providing an appropriate amount of hardware resources is another important point of focus as the CPU, RAM, and storage are important aspects of querying operations. The database server contains 4 cores, 12GB of RAM, and 64GB of disk storage to store the entire IMDB dataset.

    \section{Results and Analysis}

\subsection{Function Testing}

For ETL scripts, since they were written in mostly pure SQL using the \verb|INSERT INTO... SELECT| syntax, the primary way it was tested is by executing the \verb|SELECT| statements inside the insert statements and looking at the output. Since a lot of time was spent analyzing the data during data profiling, the outputs of the scripts were able to be validated based on knowledge gained from the data profiling phase.

For our OLAP scripts, we tried to be more systematic with our approach since these select queries could get a little convoluted (partly because we were aiming to optimize their performance). To wrap our heads around what we were doing, we broke up the queries sequentially and executed subqueries (and CTEs) individually. By verifying the output of each of the parts of the OLAP scripts, we were able to ascertain any issues with our logic that might've been present within all the mound of joins.

In the case of half of the OLAP scripts, we also crafted equivalent (but less efficient) versions of the queries to make sure our selections were correct. If the outputs of our optimized queries were exactly the same as their counterparts, then certainly our logic was correct. In that regard, these acted more like sanity checks and greatly aided the debugging process for our OLAP scripts.


\subsection{Performance Testing}

For our performance testing, the main approach was to use various hardware and database configurations, and execute all 7 of our OLAP operations on them to see how it would affect the performance (See Appendix A).

To ensure the integrity of our results, we ran our queries three times to negate any variations due to caching behavior. We then recorded the next 5 execution times in milliseconds, then got the average of those 5 execution times. We ran these tests on the final data warehouse we created (which was about 13 GB large). The \verb|EXPLAIN ANALYZE| command was also used to perform analyses on these queries; however, for the statistical query, a different approach was necessary as it was recursive in nature (which prevented the \verb|EXPLAIN ANALYZE| from properly interpreting it):

\begin{lstlisting}
SELECT TRUNCATE(TIMER_WAIT/1000000000,6) AS Duration 
  FROM performance_schema.events_statements_history 
  WHERE TIMER_START = (
    SELECT MAX(TIMER_START) 
      FROM performance_schema.events_statements_history
  )
\end{lstlisting}

In the course of running these queries repeatedly, we found that the control group (those with the default CPU and memory settings) performed more or less similarly to those queries in setups with throttled CPU and decreased memory limits. Those two factors were not as significant as we thought in affecting query performance (assuming the specs of the server, which are notated below). Nevertheless, slight differences still arose between the execution times of these three setups, and the most noticeable of these happened when executing queries on the core-limited CPU. We posit that hyperthreading across multiple CPU cores may serve to degrade the performance of some queries (while speeding up the execution of others). These variations existed on a case-to-case basis, but it seems it has something to do with how the query is parallelized under the hood (which we unfortunately have no idea how to analyze).

Aside from these two comparisons, we also ran tests on a simulated slow-down of disk speed and found that query performance was significantly impacted by this change. This makes sense, given that all our select queries involve disk reads, and prior to caching to memory a lot of disk IO operations are executed.

But overall, beyond these hardware considerations (and the query optimizations of the previous sections), we spent a long time designing the structure of the warehouse to account for ease of access and scalability. Because of the thoroughness of thought placed into the design, we noticed that we mostly only needed to perform joins on primary keys, and additional indices were not that necessary (except for certain columns such as revenue and rating, both from their respective fact tables: however, these were primarily added because of the statistical query).

\begin{verbatim}
CONTROL:
CPU: Cores 4
RAM: 12 GB (12288MB)
Disk: Unlimited bandwidth

SLOW-CPU:
CPU: Cores 1
RAM: 12 GB (12288MB)
Disk: Unlimited bandwidth

LOW-RAM:
CPU: Cores 4, CPU limit unlimited
RAM: 2 GB (12288MB)
Disk: Unlimited bandwidth

SLOW-DISK:
CPU: Cores 4, CPU limit unlimited
RAM: 12 GB (12288MB)
Disk: 100MB/s READ
\end{verbatim}

    \section{Conclusion}

In this project, we have created an OLAP application using database warehouse techniques such as ETL, indexing, query optimization and presented data in a manner that is easily
appealable to clients or users that want to gain a deeper understanding as to how to a movie's rating or revenues tie in to several different factors such as release dates, directors, 
and genres. \\

The importance of building a database warehouse is to make data-driven decisions through thorough analysis. Additionally, it is also used to store and keep historical data such as the
\texttt{FactRatingSnapshot} table which contains historical ratings of each and every movie and TV show within IMDB. Using this data warehouse, we are able to preserve history and
the moments that get snapshotted between the data. A data warehouse is not just a database for pure analysis, it's also important to remember that it contains history as we know it. \\

Preparing the data warehouse would not have been possible without our ETL script. The ETL script first extracts data from its sources periodically to update new information into the data
warehouse. Following the first step, the script then transforms the data to data that conforms to the schema and purpose of the OLAP application. Doing this enables us to keep our
analytical queries functioning to allow us to continuously update the data we retrieve to make sure our analysis is always up to date and that we are able to store the data properly.
Lastly, the last step involves the loading of the data into the warehouse itself in order to preserve the data to be stored and used in the application itself. \\

The purpose of an OLAP is to be able to store and analyze data. An OLTP is a venue of retrieving, updating, and displaying data that should be quick and efficient. On the other hand, the
OLAP extracts and uses the data from an OLTP to make data-driven decisions. These data-driven decisions are used to help companies and large corporations make smart and informed decisions
that allows their companies to perform large-scale operations and develop important business strategies. \\

Additionally, these decisions need to be made quickly as time is money for these large companies. This brings us to the important of query optimization as being able to analyze data
efficiently and effectively will put a corporation's business far ahead of its competitors. Therefore, query optimizations such as database design, indexes, query restructuring, and
even optimizations at the hardware level are all optimization strategies that these companies use to analyze their data as quickly as possible. \\

However, not all these optimization strategies can be used without thought. For example, indexes should be used if the column is used throughout multiple queries and is used as a basis
for rows in GROUP BY, ORDER BY, JOIN, and WHERE operations. Using indexes carelessly may cause for concern as retrieving new data will cause the data warehouse to consistently update
its index and would even detriment the storage space of the data warehouse. \\

While doing this project, our group had learnt and experienced multiple different aspects of database warehouse concepts and the deeper theory behind it. These include preparing an ETL
script, migrating millions of rows on data, making database design decisions based on data profiling and domain knowledge, understanding how CPU cores work during reading and writing data,
making readable and highly optimized queries, and so much more. Due to the develop of our application, we can now provide our users valuable insights and information as to what defines a
movies rating and revenue whether it be the release date, genre, or director associated with the film. To other database developers, we set an example as to that even students are able to
transform a large scale database like IMDB's into our own OLAP application that we can gain valuable insights out of. 
    
    \input{chapters/8-references}

    \section{Declarations}

\subsection{Declaration of Generative AI in Scientific Writing}

During the preparation of this work, the authors used GitHub Copilot and Claude AI to assist with the following tasks: code completion and syntax suggestions during SQL script development, debugging MySQL configuration issues, generating boilerplate code for the Next.js OLAP application, and structuring sections of the technical report. After using these tools, the authors reviewed and edited the content as needed and take full responsibility for the content of this publication.

\subsection{Record of Contribution}

\textbf{Clarence Ivan Ang}.  Developed the OLAP application frontend using Next.js and React, contributed with Malks Mogen David with designing the analytical modules for the OLAP application, and wrote the Introduction, OLAP Application, and Query Processing sections of the technical report.

\textbf{Clive Jarel Ang}.  Designed the staging area and data warehouse, implemented the dimensional model structure including fact and dimension tables, developed the ETL transformation logic for IMDb datasets, optimized MySQL configuration for large-scale data loading, and wrote the Introduction, Data Warehouse, and ETL Script sections of the technical report.

\textbf{Malks Mogen David}.  Developed the OLAP application backend using Next.js and React, contributed with Clarence Ivan Ang with designing the analytical modules for the OLAP application, and wrote the OLAP Application and Query Processing sections of the technical report.

\textbf{Rinaldo Adelrico Lee}.  Contributed to the overall system architecture design, implemented the data validation and quality checks, conducted performance testing and optimization of ETL processes, and wrote the Results and Analysis, and Conclusion sections of the technical report.

\bibliographystyle{ACM-Reference-Format}
\bibliography{sources}

\appendix

    \newpage
\appendixpage{A. Performance Testing Results}

\begin{table}[h]
    \centering
    \begin{tabular}{c|ccccccc}
    & Query 1 & Query 2 & Query 3 & Query 4 & Query 5 & Query 6 & Query 7 \\
    Control, Index & 912.6 & 1243 & 6062.2 & 2129 & 1731.4 & 2,503.60 & 1306.8 \\
SlowDisk, Index & 1984 & 1237.4 & 15251.2 & 1871.2 & 1343 & 2,974.60 & 2120.8 \\
Control, NoIndex & 1099.6 & 1283.6 & 9037.6 & 2456.8 & 1758.4 & 5,323.60 & 1303.2 \\ 
SlowCPU, NoIndex & 1083 & 1252.6 & 9308.2 & 1897 & 1380 & 5,402.40 & 1752 \\ 
LowMem, NoIndex & 1147.2 & 1255.4 & 9272 & 2131 & 1757.2 & 5,229.20 & 1316.6 \\ 
SlowDisk, NoIndex & 2463.8 & 1251 & 22298 & 1878.4 & 1392.2 & 14,706.20 & 1922

    \end{tabular}
    \caption{Query Times (ms)}
    \label{tab:my_label}
\end{table}

\end{document}