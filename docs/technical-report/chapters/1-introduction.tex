\section{Introduction}

The entertainment industry generates vast amounts of data on audience reception and financial performance. IMDb, one of the largest film and television databases, catalogs over 11.9 million titles spanning movies, TV series, streaming content, and other media formats. In addition to this, Box Office Mojo tracks daily domestic box office revenue for thousands of theatrically released films. However, these datasets remain in their original operational formats, making it difficult to perform integrated analyses that combine critical reception with commercial success. This limitation restricts the ability of industry analysts and researchers to gain deeper insights into entertainment trends and patterns.

This project addresses this gap by designing a data warehouse that integrates IMDb's non-commercial datasets with Box Office Mojo revenue data. The warehouse transforms raw source files into a structured framework optimized for Online Analytical Processing (OLAP). OLAP refers to software tools that enable interactive examination of multidimensional data through operations like drill-down (navigating from summary to detail), roll-up (aggregating to higher levels), and slicing and dicing across dimensions (Codd et al., 1993). The resulting system supports analyses of audience rating trends over time, correlations between IMDb ratings and box office performance, and genre-based popularity patterns.

The data warehouse was built using a hybrid star-snowflake architecture, also known as a starflake schema. This design balances query performance with data normalization by using bridge tables for many-to-many relationships while denormalizing attributes with bounded cardinality (Kimball \& Ross, 2013). The warehouse contains two fact tables—\texttt{FactRatingSnapshot} for audience ratings and \texttt{FactBoxOfficeRevenue} for box office data—along with five dimension tables including \texttt{DimTitle}, \texttt{DimPerson}, \texttt{DimEpisode}, and \texttt{DimDate}. Both fact tables share conformed dimensions, which would allow integrated analyses across different metrics.

An interactive web-based OLAP application provides a way to perform data analytics to the warehouse. Built using Next.js and React, the application offers five analytical modules: audience analytics, genre analysis, temporal trends, rating correlations, and revenue analysis. Users can perform drill-down operations (e.g., from annual ratings to monthly ratings for a specific genre), roll-up aggregations (e.g., from daily box office to quarterly revenue), and filtering across multiple dimensions. Overall, the application is intended for two primary user groups: entertainment industry analysts making business decisions and academic researchers investigating media trends.
